\usepackage{geometry}
\geometry{a4paper, top=40mm, left=30mm, right=30mm, bottom=35mm}

\setlength\parindent{0pt}
\usepackage[german]{babel}
\usepackage[utf8]{inputenc}


\usepackage{multicol} % Spalten
\usepackage{color} % Farben
\usepackage[hyphens]{url} % Internetadresse (mit automatischer Trennung)
\usepackage{enumitem} % Aufzählungen


% Literaturverzeichnis
\usepackage{csquotes}
\usepackage[
	sorting = none]{biblatex}
\addbibresource{../Allgemeines/Literatur.bib}


% Mathesymbole
\usepackage{amsmath, amsthm, amssymb}
\usepackage{bm} % fette Schrift in Matheumgebung
%\usepackage{physics} % Derivate richtig schreiben
%\usepackage[version=4]{mhchem} % Chemische Elemente
\usepackage[
	locale=DE,
	separate-uncertainty=true, % Fehler mit ±
	per-mode=symbol-or-fraction, % m/s im Text, sonst \frac
	% alternativ:
	% 	per-mode=reciprocal,
	% 	m s^{-1}
	output-decimal-marker=., % . statt , für Dezimalzahlen
	alsoload=synchem, % für \torr und andere chemische sachen
	]{siunitx} % si-Einheiten
	
% Chemiesymbole
\usepackage[version=3]{mhchem}

%Einfach Ableitungen etc.
\usepackage{physics}


% Bilder
\usepackage{caption}
\usepackage{graphicx, wrapfig}
\usepackage{subcaption} % Bilder in Gruppe einzeln benennen
\usepackage{tikz}
\usetikzlibrary{matrix} % zeigt Koordinatensystem an
\usepackage[european]{circuitikz} % Schaltkreise
\usetikzlibrary{arrows}
\newcommand{\mymeter}[2] % Option um Schaltsymbole zu drehen
{  % #1 = name , #2 = rotation angle
	\begin{scope}[transform shape,rotate=#2]
		\draw[thick] (#1)node(){$\mathbf V$} circle (11pt);
		\draw[rotate=45,-latex] (#1)  +(-17pt,0) --+(17pt,0);
	\end{scope}
}

\usepackage[
labelfont=bf,        % Tabelle x: Abbildung y: ist jetzt fett
font=small,          % Schrift etwas kleiner als Dokument
width=0.9\textwidth, % maximale Breite einer Caption schmaler
]{caption}
\sisetup{table-format=1.2}
\usepackage{booktabs}

% Kopf- und Fußzeile
\usepackage{fancyhdr}
\pagestyle{fancy}
\renewcommand{\sectionmark}[1]{\markright{#1}}
\renewcommand{\subsectionmark}[1]{\markright{#1}}
\fancyhead{} % Default-Einstellungen im Header löschen
\fancyhead[L]{\sc{Versuch \V}}
\fancyhead[R]{\sc{\rightmark}}


