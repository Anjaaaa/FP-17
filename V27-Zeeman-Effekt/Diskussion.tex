Der theoretische Wert für den Landé-Faktor beim anomalen Zeemann-Effekt wird durch Mittelwertbildung der Werte für $g_{12}$ in Tabelle \ref{tab:g12} berechnet. Da im Experiment nur absolute Werte bestimmt werden können, werden auch für den Mittelwert nur absolute Werte verwendet. Außerdem wird der der Übergang von $m_1=0$ nach $m_2=0$ vernachlässigt, da er verboten ist \cite{Protokoll}.

Insgesamt sind alle Werte etwas zu hoch, wobei die Werte für den normalen Zeeman-Effekt und für den $\pi$-Übergang beim anomalen Zeeman-Effekt fast im Toleranzbereich der Unsicherheit liegen. Bei der Kalibrierung des Magnetfeldes wurde nicht immer sorgfältig darauf geachtet, dass die Hall-Sonde senkrecht zum Feld steht, sodass die gemessenen Werte unter dem tatsächlichen Wert liegen. Nach \eqref{eq:Lande} sorgt das für einen zu großen Landé-Faktor. Wenn die $\sigma$-Übergänge, die von $m_1=0$ ausgehen, stark unterdrückt sind, müssten sie bei der oben erklärten Mittelwertbildung mit einem kleinen Gewicht versehen werden, sodass der Theoriewert eigentlich näher am gemessenen Wert liegt.
\begin{table}
	\centering
	\begin{tabular}{l|ccc}
		\toprule
		& Theorie & Experiment & Abweichung \\
		\midrule
		normaler Zeeman-Effekt & 1 & \SI{1.12+-0.08}{}
 & $+\SI{12}{\%}$ \\
		anomaler Zeeman-Effekt $\pi$ & 0.5 & \SI{0.57+-0.06}{}
 & $+\SI{14}{\%}$ \\
		anomaler Zeeman-Effekt $\sigma$ & 1.75 & \SI{2.1+-0.2}{}
 & $+\SI{20}{\%}$ \\
		\bottomrule
	\end{tabular}
\caption{Vergleich der Messdaten mit den Theoriewerten}
\label{tab:Vergleich}
\end{table}