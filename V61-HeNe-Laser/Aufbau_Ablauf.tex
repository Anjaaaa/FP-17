\section{Versuchsaufbau}
Der erste Versuchsteil besteht darin einen funktionsfähigen Laser aufzubauen. Zunächst befinden sich auf der optischen Schiene nur der Justierlaser an einem Ende und ein Fadenkreuz direkt vor ihm. Der Aufbau kann folgendermaßen strukturiert werden:
\begin{enumerate}
	\item Ein nicht vollständig reflektierender Spiegel wird in einigem Abstand auf der Schiene befestigt. Auf dem Fadenkreuz sind nun zwei Punkte zu sehen. Der stärker definierte wurde von der dem Laser zugewandten Seite reflektiert, er wird durch Justage des Lasers in die Mitte des Fadenkreuzes verschoben.
	\item Ein zweiter, vollständig reflektierender, Spiegel wird zwischen Justierlaser und erstem Spiegel auf die Schiene gestellt. Auf dem Fadenkreuz sind jetzt zwei weitere Punkte zu erkennen. Dieses Mal wird der unschärfere Punkt in die Mitte des Fadenkreuzes verschoben, da der Versatz des Stahlengangs zum reflektierenden Punkt hier geringer ist als bei dem scharfen Punkt.
	\item Die Kammer mit dem Lasermedium wird auf die Schiene zwischen die Spiegel gestellt. Durch die vier Justierschrauben an der Kammer kann der Strahlgang optimiert werden bis die Lasertätigkeit einsetzt.
	\item Mit Hilfe einer Photodiode und/ oder eines optischen Schirms wird die Laserleistung, durch Verstellen aller Justierschrauben, maximiert.
\end{enumerate}

\section{Messungen}
Als erstes wird die \textbf{Intensitätsverteilung zweier Moden} vermessen. Dafür wird eine vergrößernde Linse in den Strahlengang gestellt und die Intensitätsverteilung mit Hilfe einer Photodiode entlang einer horizontalen Linie vermessen. Für die $\text{TEM}_{00}$-Mode bleibt der Strahlengang ansonsten unberührt. Um die $\text{TEM}_{10}$-Mode beobachten zu können wird zusätzlich ein Wolframdraht in den Resonator gestellt. Dieser sorgt dafür, dass die zentrierte $\text{TEM}_{00}$-Mode unterdrückt wird. \\
Danach wird mit Hilfe eines Polarisators hinter dem auskoppelnden Spiegel die \textbf{Polarisation} des Lichts bestimmt. Hierfür wird der Polarisator in kleinen Winkeln gedreht und die jeweilige Intensität an der Photodiode aufgenommen. \\
Die \textbf{Wellenlänge} des Lichts wird bestimmt indem ein Gitter in den Strahlengang gestellt wird und dann an einem entfernten Schirm die Abstände der Intensitätsmaxima zum Hauptmaximum gemessen werden. \\
Zuletzt wird die \textbf{Stabilitätsbedingung} \eqref{eq:Stabilitat} für einen Resonator mit zwei konkaven Spiegeln und einen Resonator mit einem konkaven und einem planaren Spiegel überprüft, indem die maximale Intensität für verschiedenen Resonatorlängen gemessen wird.