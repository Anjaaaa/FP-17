\subsection{Polarisation}
Die Winkel mit den zugehörigen Intensitäten sind in Tabelle \ref{tab:Polarisation} aufgetragen. Die Funktion
\begin{align*}
	I_\text{fit} = I_0\sin^2\left(\omega\varphi + \varphi_0\right)
\end{align*}
wird an die Messwerte gefittet. Dabei ergeben sich die Parameter
\begin{align}
	I_0 &= \SI{94+-1e-6}{\micro\ampere}
 \\
	\omega &= \SI{1.013+-0.006}{}
 \\
	\varphi_0 &= \SI{0.44+-0.02}{}
 \ .
\end{align}
Die Messwerte und die gefittete Funktion sind in Abbildung \ref{fig:fitPol} zu sehen. Die Maxima befinden sich bei \SI{0.3\pi}{}
- bzw. $1.3\pi$, d.h. der Laser war in diese Richtung polarisiert.
\begin{figure}[h!]
	\centering
	\includegraphics[width=.7\textwidth]{Fit_Polarisation.png}
	\caption{Fit zur Bestimmung der Polarisationsrichtung}
	\label{fig:fitPol}
\end{figure}
\begin{table}
    \centering
    \caption{Intensität in Abhängigkeit des Winkels des Polarisators}
    \label{tab:Polarisation}
    \sisetup{parse-numbers=false}
    \begin{tabular}{
	S[table-format=1.1]
	S[table-format=3.1]
	}
	\toprule
	{$\varphi$}		& {$I \ \mathrm{in} \ \si{\micro\ampere}$}		\\ 
	\midrule
    2.62 & 0   \\
2.79 & 2   \\
2.97 & 11  \\
3.14 & 23  \\
3.32 & 42  \\
3.49 & 55  \\
3.67 & 67  \\
3.84 & 80  \\
4.01 & 89  \\
4.19 & 99  \\
4.36 & 99  \\
4.54 & 86  \\
4.71 & 78  \\
4.89 & 59  \\
5.06 & 40  \\
5.24 & 25  \\
5.41 & 9   \\
5.59 & 3   \\
5.76 & 0   \\
5.93 & 2   \\
6.11 & 9   \\
0.00 & 16  \\
0.17 & 31  \\
0.35 & 45  \\
0.52 & 60  \\
0.70 & 74  \\
0.87 & 87  \\
1.05 & 102 \\
1.22 & 79  \\
1.40 & 82  \\
1.57 & 70  \\
1.75 & 62  \\
1.92 & 43  \\
2.09 & 26  \\
2.27 & 16  \\
2.44 & 5   \\

    \bottomrule
    \end{tabular}
    \end{table}
