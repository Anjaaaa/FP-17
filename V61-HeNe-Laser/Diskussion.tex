In den Plots zur Intensitätsverteilung der beiden TEM-Moden ist zu sehen, dass die Intensität jeweils zu einer Seite verschoben ist, anstatt symmetrisch zu sein. Die gemessene Intensitätsverteilung passt zudem nicht zur optisch beobachteten. \\
Die Vermessung der Polarisationsrichtung ergibt wie erwartet eine Periodizität von $\pi$. Die Abweichung von $\omega$ vom erwarteten Wert $\omega_\text{exp}=1.000$ beträgt +\SI{1.3}{\%}. Diese kleine Abweichung kann mit der menschlichen Komponente der Messung erklärt werden. \\
Die Messung der Wellenlänge ergibt einen Wert, der \SI{0.7}{\%} höher ist als der Literaturwert in \cite{V61}. \\
Die Stabilität konnte nur bedingt überprüft werden, da nicht alle möglichen Resonatorlängen vermessen werden konnten. Im Falle der beiden gekrümmten Spiegel konnte die Stabilitätsbedinung im Bereich der Möglichkeiten bestätigt werden. Im Fall des Resonators mit einem planaren Spiegel dagegen gab es eine obere Grenze von \SI{0.98}{\metre} für den Aufbau eines funktionierenden Lasers, der signifikant unter dem theoretisch vorhergesagten Wert von \SI{1.40}{\metre} bleibt. \\
Da Messfehler durch kleinschrittige Datennahme (Polarisation) oder Aufnahme mehrerer Daten für die Berechnung eines Wertes (Wellenlänge) so gut wie möglich vermieden wurden, liegt die Ursache für die Abweichungen vermutlich in Verunreinigungen der optischen Bauteile. Einzig bei der Überprüfung der Stabilitätsbedingung war die menschliche Komponente wahrscheinlich entscheidend, denn die Einstellung eines Lasers erfordert bei einem planaren Spiegel größere Sorgfalt als bei einem gekrümmten. Das liegt daran, dass das Licht exakt senkrecht auf den planaren Spiegel fallen muss und keine Fokussierung stattfindet. Da die erforderliche Genauigkeit mit dem Resonatorabstand steigt, lässt sich die obere Grenze mit fehlender Geduld und feinmotorischen Fähigkeiten der Experimentatoren erklären.