 \subsection{Berechnung der Wellenlänge}
 Die Wellenlänge des HeNe-Lasers wird mit Hilfe eines Interferenzbildes berechnet, das durch die Beugung am Gitter entsteht. Jeweils rechts und links vom Hauptmaximum (Maximum 0-ter Ordnung) werden die Abstände $a$ der Maxima $n$-ter Ordnung zum Hauptmaximum gemessen. Die Wellenlänge $\lambda$ lässt sich dann mit dem Abstand $b$ zwischen Gitter und Schirm berechnen nach:
 
 \begin{equation}
 \lambda = \frac{g \cdot \sin{\phi_n}}{n} \quad .
 \end{equation}
 
Dabei ist $g$ die Gitterkonstante und $\phi_n$ der Winkel zum $n$-ten Maximum.
Es gilt
\begin{align}
\phi_n &= \arctan{\left( \frac{a}{b} \right)} \quad \textrm{,}\\
g &= \SI{100}{\per\milli\meter} \quad \textrm{,}\\
b &= \SI{104}{\centi\meter} \quad .
\end{align}
Die Wellenlänge für das einzelne Maximum ist jeweils aus dem Mittelwert des rechten und linken Abstandes berechnet (siehe Tabelle~\ref{tab:Wellenlaenge}). Daraus ergibt sich eine Wellenlänge von
\begin{equation}
\lambda = \SI{637.0+-1.4}{\nano\meter}
 \quad .
\end{equation}
\begin{table}
    \centering
    \caption{Berechnung der Wellenlänge $\lambda$ durch die Abstände der Interferenzmaxima}
    \label{tab:Wellenlaenge}
    \sisetup{parse-numbers=false}
    \begin{tabular}{
	S[table-format=1.0]
	@{${}\pm{}$}
	S[table-format=2.2, table-number-alignment = left]
	S[table-format=2.2]
	S[table-format=3.2]
	@{${}\pm{}$}
	}
	\toprule
	\multicolumn{2}{c}{Ordnung Maximum}		& {Abstand rechts}		& 
	\multicolumn{2}{c}{Abstand links}		& {$\lambda$ in $\si{\nano\meter}$}		\\ 
	\midrule
    1 & 6.70  & 6.70  & 642.90 \\
2 & 13.30 & 13.35 & 635.43 \\
3 & 20.10 & 20.30 & 635.56 \\
4 & 26.90 & 27.50 & 632.57 \\
5 & 34.80 & 35.30 & 638.74 \\
6 & 42.10 & 43.90 & 636.82 \\

    \bottomrule
    \end{tabular}
    \end{table}
