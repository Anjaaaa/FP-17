Bei dem Aufbau des Versuchs wird besonders ein optisches Bauteil hervorgehoben. Die Lummer-Gehrcke-Platte erlaubt eine genaue Auflösung der Spektrallinien. Danach wird die Durchführung des Experiments erklärt.
\subsection{Versuchsaufbau}
Der Versuchsaufbau ist in Abbildung \ref{fig:abb7} dargestellt. Eine mit Cadmium-Gas gefüllte Spektrallampe befindet sich in einem Magnetfeld, das durch zwei Spulen erzeugt wird und dessen Magnetfeldstärke veränderbar ist. Der Strahlengang wird durch ein Obektiv und eine Linse fokussiert und mit Hilfe eines Spalts möglichst parallel zur optischen Achse eingestellt. Ein Geradsichtprisma spaltet die Spektrallinien auf. Ein Polarisationsfilter ermöglicht es longitudinale und transversale Komponenten des Lichtes einzeln zu betrachten. Ein zweiter Spalt dient dazu, eine Spektrallinie auszuwählen. Abermals fokussieren zwei Linsen den Strahl, damit der Strahl mit einer möglichst genauen Auflösung der Spektrallinien auf dem zweiten Spalt und dann auf die Lummer-Gehrcke-Platte fällt. Die Lummer-Gercke-Platte (siehe Kapital \ref{sec:platte}) erhöht die Auflösung des Strahls. Mit einer Kamera kann das Bild aufgenommen werden, dass durch die Lummer-Gehrcke-Platte entsteht.

\begin{figure}
	\includegraphics[width = \textwidth]{Abb7.pdf}
	\caption{Versuchsaufbau \cite{\V}}
	\label{fig:abb7}
\end{figure}
\clearpage
\subsubsection{Lummer-Gehrcke-Platte}\label{sec:platte}
Die Lummer-Gehrcke Platte ist eine dünne Schicht aus Glas, die durch Totalreflexion und konstruktive Interferenz die Auflösung der Wellenlänge vergrößert (siehe Abb.~\ref{fig:abb8}). Um konstruktive Interferenz zu erhalten muss die Bragg-Bedingung 
\begin{align}
	2d\cos(\alpha) = n \lambda \quad, n \in \mathbb{N}
\end{align}
erfüllt sein. Die technischen Daten der in diesem Versuch verwendeten Lummer-Gehrcke-Platte sind:
\begin{align}
	d &= \si{4}{\milli\meter} \notag \\
	L &= \si{120}{\milli\meter} \notag
\end{align}
Daraus ergibt sich eine minimale Wellenlängendifferenz, die gerade noch aufgelöst werden kann:
\begin{align}
	\Delta \lambda_D &= \frac{\lambda^2}{2 d} \sqrt{\frac{1}{n^2 - 1}} \\
	\Delta \lambda_{D, \textrm{red}} &= \SI{4.89e-11}{\meter} \notag \\
		\Delta \lambda_{D, \textrm{blue}} &= \SI{2.7e-11}{\meter} \notag
\end{align}
Als Brechungsindex wird für die Rechnung angenommen: 
\begin{align}
n_{\textrm{red}}(\lambda =\SI{644}{\nano\meter}) &= \num{1.4567}  \notag \\
n_{\textrm{blue}}(\lambda = \SI{480}{\nano\meter}) &= \num{1.4635} \quad . \notag
\end{align}
Das Auflösungsvermögen der Lummer-Gehrcke-Platte ist:
\begin{align}
	A(\lambda) &= \frac{L}{\lambda} \left(n^2 -1  \right) \\
	A(\lambda_{\textrm{red}}) &= \num{2.09e5} \notag \\
	A(\lambda_{\textrm{blue}}) &= \num{2.85e5} \notag
\end{align}
Die Lummer-Gehrke Platte erreicht somit eine recht hohe Vergrößerung.
\begin{figure}	\includegraphics[width = 0.7\textwidth]{Abb8.pdf}
	\caption{Lummer-Gehrcke Platte \cite{\V}}
	\label{fig:abb8}
\end{figure}

\subsection{Durchführung}
Zunächst wird der \textbf{Strahlengang justiert}. Die Linsen werden so verschoben, dass der Strahl gut auf das Prisma, die beiden Spalte und die Lummer-Gehrke-Platte auftrifft. \\
Sobald der Strahlengang gut eingestellt ist, kann das \textbf{Magnetfeld hochgefahren} werden, damit der Zeemann-Effekt eintritt und die Energieniveaus aufspalten. \\
Insgesamt werden fünf \textbf{Bilder aufgenommen}. Für die blaue Spektrallinie macht man ein Foto ohne Magnetfeld, eines mit Magnetfeld und einem Polarisationsfilter in horizontale Richtung und eines mit Magnetfeld und einem Polarisationsfilter in vertikale Richtung. Für die rote Spektrallinie ist das Bild ohne Magnetfeld identisch zu dem Bild mit Magnetfeld und vertikalem Polarisationsfilter, da der $\pi$-Übergang beim normalen Zeemann-Effekt keine Aufspaltung zeigt. Deswegen werden für die rote Spektrallinie nur zwei Bilder benötigt. Für die Bilder mit Magnetfeld wird der Regelstrom, der an der Spule anliegt, jeweils so weit aufgedreht, dass eine Aufspaltung erkennbar ist, die Linien aber nicht verschwimmen. \\
Als letztes wird das \textbf{Magnetfeld geeicht}. Dazu wird mit einer Hallsonde die Magnetfeldstärke in Abhängigkeit des Regelstroms gemessen. Es ist wichtig sowohl bei ansteigendem als auch bei abfallendem Strom zu messen. So entsteht eine Hysterese-Kurve.