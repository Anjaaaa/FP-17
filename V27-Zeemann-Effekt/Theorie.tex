Um den Zeemann-Effekt zu erklären, wird hier zunächst auf die Drehbewegungen der Elektronen in einem Atom eingegangen, die zu magnetischen Momenten führen und schlussendlich die Aufspaltung der Energieniveaus in einem Magnetfeld bewirken. Dann wird ausführlich darauf eingegangen, was beim Zeemann-Effekt beobachtet werden kann. Die theoretischen Grundlagen werden am Beispiel von Cadmium erklärt, dass in diesem Versuch verwendet wird.

\subsection{Magnetisches Moment von einem Atom}
Der gesamte Drehimpuls $\vec{j}$ eines einzelnen Hüllenelektrons setzt sich aus dem Spin $\vec{s}$ und dem Bahndrehimpuls $\vec{l}$ zusammen. Ein elektrischer Kreisstrom führt zu einem magnetischen Moment. Das Elektron hat ein Magnetisches Moment, das vom Spin generiert wird, $\vec{\mu}_s$ und eines das vom Bahndrehimpuls herrührt $\vec{\mu}_l$. Beide Magnetischen Momente zeigen entgegengesetzt zum entsprechenden Drehimpuls und sind proportional zum Betrag der Drehimpulse. Der Betrag vom Spin und vom Bahndrehimpuls kann durch die diskreten Quantenzahlen $s$ und $l$ ausgedrückt werden.
\begin{align}
	| \vec{s} | &= \sqrt{s(s+1)} \hbar \\
	| \vec{l} | &= \sqrt{l(l+1)} \hbar
\end{align}
Die magnetischen Momente sind 
\begin{align}
	\vec{\mu}_s &= - g_s \mu_B  \sqrt{s(s+1)} \vec{e}_s \\
	\vec{\mu}_l & = - \mu_B  \sqrt{l(l+1)} \vec{e}_l 
\end{align}
mit den Einheitsvektoren $\vec{e}_s$ und $\vec{e}_l$, dem Landé-Faktor $g_s$ und dem Bohrschen Magneton \\
\begin{equation}
	\mu_B =- \frac{e_0 \hbar}{2 m_0}  \quad .
\end{equation}
Gibt es nun mehr als ein Hüllenelektron, so wechselwirken die Spins und Bahndrehimpulse und addieren sich zu einem Gesamtdrehimpuls $\vec{J}$ aller Hüllenelektronen. Bei Atomen mit niedriger Kernladungszahl gilt die $LS$-Kopplung
\begin{align}
	\vec{J} = \vec{L} + \vec{S} = \sum_i \vec{l}_i + \sum_i \vec{s}_i \quad .
\end{align}
Wenn die Kernladungszahl groß ist, dominiert die Wechselwirkung zwischen den einzelnen Spins und Bahndrehimpulsen. Es gibt keinen Gesamtspin oder Gesamtbahndrehimpuls mehr, sonder nur noch einen Gesamtdrehimpuls. Man spricht von der $j$-$j$-Kopplung mit
\begin{align}
	\vec{J} = \sum_i \vec{j}_i = \sum_i \left( \vec{l}_i + \vec{s}_i \right) \quad .
\end{align}
Bei Atomen mit mittlerer Kernladungszahl mischen sich die beiden Effekte. In diesem Versuch wird Cadmium untersucht. Cadmium hat die Kernladungszahl $Z = 48$ und fällt unter die $LS$-Kopplung. Um das gesamte magnetische Moment zu berechnen ist nur die nicht abgeschlossene Elektronenhülle relevant, da abgeschlossene Schalen keinen Gesamtdrehimpuls haben. Das gesamte magnetische Moment setzt sich also aus dem magnetischen Moment des Gesamtspins und des Gesamtbahndrehimpulses zusammen
\begin{align}
	\vec{\mu} = \vec{\mu}_L + \vec{\mu}_S  \quad .
\end{align}
Das magnetische Moment $\vec{\mu}$ ist wegen des Lande-Faktors des Spins nicht mehr parallel zum Gesamtdrehimpuls $\vec{J}$. Es ist aber die Projektion von $\vec{\mu}$ auf $\vec{J}$ relevant. Geometrische Überlegungen ergeben dann 
\begin{align}
	\vec{\mu}_J = \mu_B g \sqrt{J(J+1)} \vec{e}_J
\end{align}
mit dem Landé-Faktor
\begin{align}
g =	\frac{3J(J+1) + S(S+1) - L(L+1)}{2J(J+1)} \quad .
\end{align}
Hierbei ist $J$ die Quantenzahl des Gesamtdrehimpulses, $L$ die des Gesamtbahndrehimpulses und $S$ die des Gesamtspins.

\subsection{Normaler Zeemann-Effekt}\label{sec:normaler}
Beim normalen Zeemann-Effekt ist der Gesamtspin $S$ null. Der Effekt beschreibt folgendes: Ein magnetisches Moment in einem Magnetfeld $\vec{B}$ hat eine Energie, wenn das magnetische Moment parallel zu den Magnetfeldlinien ausgerichtet ist oder zumindest parallele Anteile hat. Die Energie ist dann
\begin{align}
	E = - \vec{\mu}_J \cdot \vec{B} = m_J g \mu_B 
\end{align}
wobei $m_J$ die Orientierungsquantenzahl von $J$ in Richtung des Magnetfeldes beschreibt. Die Orientierungsquantenzahl kann Werte von $J$ bis $-J$ annehmen. Es Erfolgt eine Aufspaltung der Energieniveaus mit den Energiedifferenzen
\begin{align}\label{eq:Energieverschiebung}
	dE = g\mu_B B \quad .
\end{align}
Der Landé-Faktor ist für alle Übergänge beim normalen Zeemann-Effekt identisch, da der Spin verschwindet und $J = L$ ist,  gilt 
\begin{align}
	g = 1 \quad .
\end{align}
Übergänge von Elektronen zwischen den Niveaus sind möglich, wenn die Übergänge die Auswahlregel $\Delta m_J = 0, \pm 1$ erfüllen. Das Elektron gewinnt durch den Übergang in ein niedrigeres Energieniveaus Energie und strahlt als schwingender Dipol eine elektromagnetische Welle mit der Frequenz $\nu= \Delta E / h $ ab. Für $\Delta m_J = 0$ strahlt das Elektron linear polarisiertes Licht ($\pi$-Übergang) ab und für $\Delta m = \pm 1$ zirkular polarisiertes Licht ($\sigma_{\pm}$-Übergang). 
Die Übergänge unterschiedlicher Polarisation können nur in bestimmten Orientierungen zum Magnetfeld beobachtet werden. Das linear polarisierte Licht ist sichtbar, wenn man den Anteil beobachtet, der senkrecht zu den Magnetfeldlinien (longitudinale Betrachtung) steht und den zirkular polarisierten Anteil entsprechend bei der Betrachtung des horizontalen Anteils (transversale Betrachtung), siehe Abb. \ref{fig:abb4}. Der Grund dafür ist die Ausrichtung des Dipols zum Magnetfeld.
\begin{figure}
	\includegraphics[width = 0.4\textwidth]{Abb4.pdf}
	\caption[Polarisationsrichtung]{Aufspaltung des Energieniveaus für $J=1$ in Abhängigkeit der Polarisationsrichtung des ausgestrahlten Lichtes \cite{\V}}
	\label{fig:abb4}
	\end{figure}
\subsubsection{Cadmium -- rote Spektrallinie $\lambda = \SI{643.8}{\nano\meter}$}
Bei dem Übergang von dem Niveau $\ce{^1D_2}$ auf das Niveau $\ce{^1P_1}$ entsteht eine rote Spektrallinie mit einer Wellenlänge von $\lambda = \SI{643.8}{\nano\meter}$. Die Quantenzahlen der Niveaus sind:
\begin{align}
	&\ce{^1D_2} : \quad L=1, J =1, S =0 \notag \\
	&\ce{^1P_1}: \quad L = 2, J = 2, S = 0 \notag
\end{align}	
Der Landè-Faktor ist immer $g=1$. Es sind neun verschiedene Übergänge möglich (siehe Abb. \ref{fig:mausi1}).
\begin{figure}
	\includegraphics[width = 0.4\textwidth]{Mausi1.png}
	\caption[Aufspaltung von $\ce{^1D_2}$ auf $\ce{^1P_1}$]{Aufspaltung mögliche Übergänge  von Niveau $\ce{^1D_2}$ auf das Niveau $\ce{^1P_1}$. \cite{V27_mausi}}
	\label{fig:mausi1}
\end{figure}

\subsection{Anormaler Zeemann-Effekt}
Der anormale Zeemann-Effekt unterscheidet sich vom normalen Zeemann-Effekt dadurch, dass der Gesamtspin $S$ von null verschieden ist. Der anormale Zeemann-Effekt tritt häufiger auf als der normale Zeemann-Effekt. Die Auswahlregel für die Übergänge $\Delta m_J = 0, \pm 1$ gilt weiterhin und auch die Polarisation der emittieren Welle ist genau wie beim normalen Zeemann-Effekt (siehe Kaptiel \ref{sec:normaler}). Die Energiedifferenzen zwischen den Niveaus sind hingegen nicht mehr äquidistant, da der Landé-Faktor der einzelnen Niveaus unterschiedlich ist. Es gilt
\begin{align}
	\Delta E &= \left( m_{J,1} g_1(L_1, S_1, J_1) - m_{J,2} g_2(L_2, S_2, J_2)  \right) \mu_B B \notag \\
	& =g_{12} \mu_B \Delta m_J
\end{align}
\subsubsection{Cadmium -- blaue Spektrallinie $\lambda = \si{480}{\nano\meter}$}
Der Übergang von dem Niveau $\ce{^3P_1}$ auf das Niveau $\ce{^3S_1}$ erzeugt eine blaue Spektrallinie mit einer Wellenlänge von $\lambda = \si{480}{\nano\meter}$. Die Quantenzahlen der Niveaus sind
\begin{align}
	&\ce{^3P_1} : \quad L=1, J =1, S =1 \notag \\
	&\ce{^3S_1}: \quad L = 0, J = 1, S = 1 \notag
 \end{align}	
Die Übergänge sind in Abbildung \ref{fig:mausi2} dargestellt und die Faktoren $g_{12}$ sind in Tabelle~\ref{tab:g12} zu finden.

\begin{figure}
	\includegraphics[width = 0.4\textwidth]{Mausi2.png}
	\caption{Mögliche Übergänge  von Niveau $\ce{^3P_1}$ auf das Niveau $\ce{^3S_1}$. \cite{V27_mausi}}
	\label{fig:mausi2}
\end{figure}

\begin{table}
\begin{tabular}{cccccc}
	& $m_1$ & $g_1$ & $m_2$ & $g_2$ & $g_12$ \\
	\hline
	$\sigma_-$ & -1 & 2& 0 & 1.5 & -2 \\
	& 0 & 2 & 1 & 1.5 & -1.5 \\
	\hline
	$\pi$ & -1 & 2 & -1 & 1.5 & -0.5 \\
	& 0 & 2 & 0 & 1.5 & 0 \\
	& 1 & 2 & 1 & 1.5 & 0.5 \\
	\hline
	$\sigma_+$ & 0 & 2 & -1 & 1.5 & 1.5 \\
	& 1 & 2 & 0 & 1.5 & 2
\end{tabular}
\caption[Anormaler Zeemann-Effekt $g_{12}$]{Berechnung der Faktoren $g_{12}$ bei dem Übergang von Niveau $\ce{^3P_1}$ auf~$\ce{^3S_1}$.}
\label{tab:g12}
\end{table}




