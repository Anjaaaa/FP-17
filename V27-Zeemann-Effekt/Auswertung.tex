\subsection{Eichung des Magnetfeldes}
Um den Zusammenhang zwischen Magnetfeldstärke und eingestelltem Strom herauszufinden, wird eine Eichmessung durchgeführt. Dabei werden zehn Stromstärken zwischen (0-15.5)\si{\ampere} eingestellt und mit einer Hall-Sonde das zugehörige Magnetfeld gemessen. Diese Messung wird einmal mit steigenden und einmal mit fallenden Stromstärken durchgeführt, um Hysterese-Effekte zu berücksichtigen. Die gemessenen Werte sind in Tabelle \ref{tab:Messwerte} aufgeführt. Bei den Stromstärken wird ein Ablesefehler von \SI{0.5}{\ampere} berücksichtigt. Aus den Quotienten der Wertepaare wird die Proportionalitätskonstante 
\begin{align}
	a = \SI{0.0572+-0.0012}{\tesla\per\ampere}

\end{align}
berechnet.
\begin{table}
    \centering
    \caption{Stromstärke $I_1,I_2$ beim Auftreten des Maximums für verschiedene Anregungsfrequenzen $\nu$}
    \label{tab:Werte}
    \sisetup{parse-numbers=false}
    \begin{tabular}{
	S[table-format=2.3]
	S[table-format=3.0]
	S[table-format=2.0]
	S[table-format=3.0]
	S[table-format=2.0]
	}
	\toprule
	{$\nu \ \mathrm{in} \ \si{\mega\hertz}$}		& {$I_1 \ \mathrm{in} \ \si{\milli\ampere}$}		& 
	{$I_2 \ \mathrm{in} \ \si{\milli\ampere}$}		\\ 
	\midrule
    10.588 & 232 & 5  & 307 & 5  & 41.5 & 0.5 \\
15.970 & 357 & 9  & 407 & 5  & 16.7 & 0.3 \\
20.560 & 453 & 9  & 546 & 4  & 18.1 & 0.3 \\
23.870 & 587 & 10 & 633 & 10 & 15.2 & 0.6 \\
29.420 & 717 & 10 & 787 & 8  & 15.3 & 0.4 \\

    \bottomrule
    \end{tabular}
    \end{table}


\subsection{Landé-Faktor}
Mit Hilfe des Bildbearbeitungsprogramm werden die Größen $\Delta s$ und $\delta s$ in den Photos ausgemessen. Diese Werte sind in den Tabellen \ref{tab:WerteFotosRot} und \ref{tab:WerteFotosBlau} zu sehen. Die Breite einer Linie ist im Bereich von 40-50 Pixeln, daher wird ein Ablesefehler von 10 Pixeln eingerechnet.


Mit Hilfe von
\begin{align}
	\Delta\lambda = \frac{1}{2}\frac{\delta s}{\Delta s}\Delta\lambda_\text{D}
\end{align}
kann die Wellenlängenverschiebung $\Delta\lambda = \lambda - \lambda_0$ berechnet werden.

Die Änderung der Energiedifferenz zwischen zwei Niveaus beim Einschalten eines Magnetfeldes, $dE$, kann durch die Wellenlänge eines beim Übergang emittierten Photons ausgedrückt werden:
\begin{align*}
	dE &= E(B\not=0) - E(B=0) \\
	&=\frac{hc}{\lambda} - \frac{hc}{\lambda_0} \approx hc\frac{\Delta\lambda}{\lambda_0^2} \ ,\quad\text{da }\lambda \approx \lambda_0 \ .
\end{align*}
Durch Gleichsetzen mit \eqref{eq:Energieverschiebung} kann so der Landé-Faktor
\begin{align}\label{eq:Lande}
	g = \frac{dE}{\mu_\text{B}B} = \frac{hc}{\mu_\text{B}B}\frac{\Delta\lambda}{\lambda_0^2}
\end{align}
berechnet werden. Neben den physikalischen Konstanten $h,c,\mu_\text{B}$, werden die Magnetfeldstärke $B$ und die Wellenlänge $\lambda_0$ benötigt:
\begin{align*}
	B_{\text{red,}\sigma}= &\SI{514(11)}{\milli\tesla} \quad && \quad &\lambda_{\text{red},0} = &\SI{643.8}{\nano\meter} \\
	B_{\text{blue,}\sigma}= &\SI{286(6)}{\milli\tesla} \quad && \quad &\lambda_{\text{blue},0} = &\SI{480.0}{\nano\meter} \\
	B_{\text{blue,}\pi}= &\SI{857(18)}{\milli\tesla} \ . && & & & \\
\end{align*}
Gemittelt ergeben sich so die Landé-Faktoren
\begin{align}
	g_{\text{red,}\sigma} &= \SI{1.12+-0.08}{}
\notag \\
	g_{\text{blue,}\sigma} &= \input{build/gBlueSigma.tex}\notag \\
	g_{\text{blue,}\pi} &= \SI{0.57+-0.06}{}
 \ .
\end{align}
Es gilt zu beachten, dass bei der roten Linie der normale Zeemann-Effekt auftritt, sodass hier der \glqq echte\grqq\ Landé-Faktor berechnet wurde. Bei der blauen Linie kann der anomale Zeemann-Effekt beobachtet werden, daher ist der berechnete Landé-Faktor die Superposition
\begin{align*}
	g_{12} = m_1g_1-m_2g_2 \ .
\end{align*}

\begin{table}
    \centering
    \caption{Abstände $\Delta s,\delta s$ in Pixel bei der roten Linie}
    \label{tab:WerteFotosRot}
    \sisetup{parse-numbers=false}
    \begin{tabular}{
	S[table-format=3.0]
	S[table-format=3.0]
	}
	\toprule
	{$\Delta s_\mathrm{red}$}		& {$\delta s_{\mathrm{red,}\sigma}$}		\\ 
	\midrule
    \input{build/tableFotosRed.tex}
    \bottomrule
    \end{tabular}
    \end{table}

\begin{table}
    \centering
    \caption{Abstände $\Delta s,\delta s$ in Pixel bei der blauen Linie}
    \label{tab:WerteFotosBlau}
    \sisetup{parse-numbers=false}
    \begin{tabular}{
	S[table-format=3.0]
	S[table-format=3.0]
	S[table-format=2.0]
	}
	\toprule
	{$\Delta s_\mathrm{blue}$}		& {$\delta s_{\mathrm{blue,}\sigma}$}		& 
	{$\delta s_{\mathrm{blue,}\pi}$}		\\ 
	\midrule
    224 & 108 & 96 \\
196 & 96  & 85 \\
188 & 88  & 78 \\
176 & 84  & 73 \\
164 & 80  & 65 \\
156 & 72  & 62 \\
152 & 72  & 61 \\
144 & 68  & 61 \\
148 & 64  & 55 \\
132 & 64  & 49 \\
132 & 60  & 47 \\
128 & 60  & 50 \\
124 & 56  & 41 \\
108 & 52  & 35 \\

    \bottomrule
    \end{tabular}
    \end{table}

