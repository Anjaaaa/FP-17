Der theoretische Wert für den Landé-Faktor beim anomalen Zeemann-Effekt wird durch Mittelwertbildung der Werte für $g_{12}$ in Tabelle \ref{tab:g12} berechnet. Da im Experiment nur absolute Werte bestimmt werden können, werden auch für den Mittelwert nur absolute Werte verwendet. Bei dieser Berechnung wird vernachlässigt, dass einige Übergänge wahrscheinlicher sind als andere, z. B. ist der $\pi$-Übergang von $m_1=0$ nach $m_2=0$ sehr unwahrscheinlich, sodass der Theoriewert in diesem Fall näher bei 0.5 liegen müsste. Dasselbe wird für den $\sigma$-Übergang von $m_1=0$ auf $m_2=\pm1$ vermutet, dieser Theoriewert läge dann näher bei 2.


Wenn die genannten Übergänge stark unterdrückt wären, wären die gemessenen Werte sehr gut, allerdings leicht zu hoch. Beim Eichen des Magnetfeldes wurde nicht immer sorgfältig darauf geachtet, dass die Hall-Sonde senkrecht zum Feld steht, sodass die gemessenen Werte unter dem tatsächlichen Wert liegen. Nach \eqref{eq:Lande} sorgt das für einen zu großen Landé-Faktor.
\begin{table}
	\centering
	\begin{tabular}{l|ccc}
		\toprule
		& Theorie & Experiment & Abweichung \\
		\midrule
		normaler Zeemann-Effekt & 1 & \SI{1.12+-0.08}{}
 & $+\SI{12}{\%}$ \\
		anomaler Zeemann-Effekt $\pi$ & 0.33 & \SI{0.57+-0.06}{}
 & $+\SI{71}{\%}$ \\
		anomaler Zeemann-Effekt $\sigma$ & 1.75 & \SI{2.1+-0.2}{}
 & $+\SI{20}{\%}$ \\
		\bottomrule
	\end{tabular}
\caption{Vergleich der Messdaten mit den Theoriewerten}
\label{tab:Vergleich}
\end{table}