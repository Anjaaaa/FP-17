\begin{table}[h!]
    \centering
    \caption{Stromstärken $I_1,I_2$ beim Auftreten des Maximums für verschiedene Anregungsfrequenzen $\nu_e$ mit dem jeweiligen Skalierungsfaktor}
    \label{tab:Werte}
    \sisetup{parse-numbers=false}
    \begin{tabular}{
	S[table-format=2.3]
	S[table-format=3.0]
	@{${}\pm{}$}
	S[table-format=2.0, table-number-alignment = left]
	S[table-format=3.0]
	@{${}\pm{}$}
	S[table-format=2.0, table-number-alignment = left]
	S[table-format=2.1]
	@{${}\pm{}$}
	S[table-format=1.1, table-number-alignment = left]
	}
	\toprule
	{$\nu_e \ \mathrm{in} \ \si{\mega\hertz}$}		& \multicolumn{2}{c}{$I_1 \ \mathrm{in} \ \si{\milli\ampere}$}		& 
	\multicolumn{2}{c}{$I_2 \ \mathrm{in} \ \si{\milli\ampere}$}		& \multicolumn{2}{c}{Skala in \si{\milli\ampere\per\centi\meter}}		\\ 
	\midrule
    10.588 & 232 & 5  & 307 & 5  & 41.5 & 0.5 \\
15.970 & 357 & 9  & 407 & 5  & 16.7 & 0.3 \\
20.560 & 453 & 9  & 546 & 4  & 18.1 & 0.3 \\
23.870 & 587 & 10 & 633 & 10 & 15.2 & 0.6 \\
29.420 & 717 & 10 & 787 & 8  & 15.3 & 0.4 \\

    \bottomrule
    \end{tabular}
    \end{table}
