Tabelle \ref{tab:Vergleich} zeigt die bestimmten Größen und vergleicht sie mit den Theoriewerten aus den angegebenen Referenzen. Die Bestimmung der Landé-Faktoren und der Kernspins ergibt sehr gute Werte, deren Vertrauensintervalle (mit Ausnahme von $g_{F2}$) den Theoriewert einschließen. Der Wert für das Erdmagnetfeld ist erstaunlich gut, wird bedacht, dass Messungen des Erdmagnetfelds üblicherweise um Größenordnungen abweichen. Das Isotopenverhältnis im Experiment entspricht nicht dem natürlichen Isotopenverhältnis. \\
Die Spin-Bahn-Wechselwirkung muss nicht berücksichtigt werden. Die Untersuchung des quadratischen Zeemann-Effekts zeigt, dass der Effekt für so kleine Magnetfelder, wie sie hier verwendet werden, keine Bedeutung hat.
\begin{table}
	\centering
	\begin{tabular}{lccccc}
		\toprule
		Größe (Referenz) & & Experiment & Theorie & Referenz & Abweichung \\
		\hline
		Landé-Faktor \eqref{eq:LandeExp}& $g_{F1}$ & \SI{0.50+-0.01}{}
 & $\nicefrac{1}{2}$ & \cite{Lande} & $\pm$\SI{0.0}{\%} \\
		& $g_{F2}$ & \SI{0.35+-0.01}{}
 & $\nicefrac{1}{3}$ & \cite{Lande} & +\SI{5.0}{\%} \\
		Erdmagnetfeld \eqref{eq:BErdeExp} & $B_\text{Erde}$ & \SI{30+-2}{\micro\tesla}
 & \SI{49}{\micro\tesla} & \cite{BErde} & \SI{-38.8}{\%} \\
		Kernspin \eqref{eq:KernspinExp} & $I_1$ & \SI{1.51+-0.06}{}
 & $\nicefrac{3}{2}$ & \cite{Ru} & +\SI{0.7}{\%} \\
		& $I_2$ & \SI{2.4+-0.1}{}
 & $\nicefrac{5}{2}$ & \cite{Ru} & \SI{-4.0}{\%} \\
		Isotopenverhältnis \eqref{eq:RatioExp} & $\nicefrac{N_{85}}{N_{87}}$ & \SI{1.5+-0.2}{}
 & 2.6 & \cite{Ru} & \SI{-42.3}{\%} \\
		\bottomrule
	\end{tabular}
\caption{Vergleich der berechneten Größen mit Theoriewerten}
\label{tab:Vergleich}
\end{table}