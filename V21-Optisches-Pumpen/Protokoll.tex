\documentclass[a4,11pt]{article}
\usepackage{geometry}
\geometry{a4paper, top=40mm, left=30mm, right=30mm, bottom=35mm}

\setlength\parindent{0pt}
\usepackage[german]{babel}
\usepackage[utf8]{inputenc}


\usepackage{multicol} % Spalten
\usepackage{color} % Farben
\usepackage[hyphens]{url} % Internetadresse (mit automatischer Trennung)
\usepackage{enumitem} % Aufzählungen


% Literaturverzeichnis
\usepackage{csquotes}
\usepackage[
	sorting = none]{biblatex}
\addbibresource{../Allgemeines/Literatur.bib}


% Mathesymbole
\usepackage{amsmath, amsthm, amssymb}
\usepackage{bm} % fette Schrift in Matheumgebung
%\usepackage{physics} % Derivate richtig schreiben
%\usepackage[version=4]{mhchem} % Chemische Elemente
\usepackage[
	locale=DE,
	separate-uncertainty=true, % Fehler mit ±
	per-mode=symbol-or-fraction, % m/s im Text, sonst \frac
	% alternativ:
	% 	per-mode=reciprocal,
	% 	m s^{-1}
	output-decimal-marker=., % . statt , für Dezimalzahlen
	alsoload=synchem, % für \torr und andere chemische sachen
	]{siunitx} % si-Einheiten
	
% Chemiesymbole
\usepackage[version=3]{mhchem}

%Einfach Ableitungen etc.
\usepackage{physics}


% Bilder
\usepackage{caption}
\usepackage{graphicx, wrapfig}
\usepackage{subcaption} % Bilder in Gruppe einzeln benennen
\usepackage{tikz}
\usetikzlibrary{matrix} % zeigt Koordinatensystem an
\usepackage[european]{circuitikz} % Schaltkreise
\usetikzlibrary{arrows}
\newcommand{\mymeter}[2] % Option um Schaltsymbole zu drehen
{  % #1 = name , #2 = rotation angle
	\begin{scope}[transform shape,rotate=#2]
		\draw[thick] (#1)node(){$\mathbf V$} circle (11pt);
		\draw[rotate=45,-latex] (#1)  +(-17pt,0) --+(17pt,0);
	\end{scope}
}

\usepackage[
labelfont=bf,        % Tabelle x: Abbildung y: ist jetzt fett
font=small,          % Schrift etwas kleiner als Dokument
width=0.9\textwidth, % maximale Breite einer Caption schmaler
]{caption}
\sisetup{table-format=1.2}
\usepackage{booktabs}

% Kopf- und Fußzeile
\usepackage{fancyhdr}
\pagestyle{fancy}
\renewcommand{\sectionmark}[1]{\markright{#1}}
\renewcommand{\subsectionmark}[1]{\markright{#1}}
\fancyhead{} % Default-Einstellungen im Header löschen
\fancyhead[L]{\sc{Versuch \V}}
\fancyhead[R]{\sc{\rightmark}}


\usepackage{floatrow}
\floatplacement{figure}{!ht}
\floatplacement{table}{!ht}





\newcommand{\Versuch}{Optisches Pumpen}
\newcommand{\Betreuer}{Tobias \textsc{Mustermann}}
\newcommand{\Tag}{12.06.17}
\newcommand{\V}{V21}
\newcommand{\Korrektur}{\ }
% Bibliographie erstellen:
% 	pdflatex Protokoll.tex
% 	biber Protokoll
% 	pdflatex Protokoll.tex
% oder einfach make ausführen


\usepackage[
%	disable,
colorinlistoftodos,
linecolor=yellow,
backgroundcolor=yellow,
textwidth=0.15\textwidth,
textsize=footnotesize
]{todonotes} % Notizen

\begin{document}
\begin{titlepage}
	\par
	\raisebox{-.5\height}{\includegraphics[height=1cm]{Logo-TUDo.png}}
	\hfill
	\raisebox{-.5\height}{\includegraphics[height=1cm]{Logo-Physik.png}}
	\par
\begin{center}
\ \\
[5.5cm]	
	\textsc{\Huge Fortgeschrittenen Praktikum SS 2017} \\
[1.5cm]
	\Huge\textbf{\Versuch} \\
[1cm]
	{\large Durchführung: \Tag} \\
	{\large \Korrektur} \\
[4.5cm]
\begin{minipage}{0.4\textwidth}
	\begin{flushleft} \large
		Anja \textsc{Beck}\textsuperscript{1} \\
		Clara \textsc{Rittmann}\textsuperscript{2}
	\end{flushleft}
\end{minipage}
\hfill
\begin{minipage}{0.4\textwidth}
	\begin{flushright} \large
		\emph{Betreuer:} \\
		\Betreuer
	\end{flushright}
\end{minipage}
\end{center}
\footnotetext[1]{anja.beck@tu-dortmund.de}
\footnotetext[2]{clara.rittmann@tu-dortmund.de}
\end{titlepage}


\tableofcontents
\clearpage



\textbf{Ziel des Versuchs} \\
Die theoretische Betrachtung und die Beschreibung von Aufbau und Durchführung orientieren sich eng an der Versuchsanleitung \cite{\V}. Alle Berechnung und Plots werden mit Python 3.6 durchgeführt bzw. erstellt, zum Fitten wird die Funktion \textit{curve\_fit} verwendet.

\section{Theorie}
Um den Zeeman-Effekt zu erklären, wird hier zunächst auf die Drehimpulse der Elektronen in einem Atom eingegangen, die zu magnetischen Momenten führen und schlussendlich die Aufspaltung der Energieniveaus in einem Magnetfeld bewirken. Dann wird ausführlich darauf eingegangen, was beim Zeeman-Effekt beobachtet werden kann. Die theoretischen Grundlagen werden am Beispiel von Cadmium erklärt, das in diesem Versuch verwendet wird.

\subsection{Magnetisches Moment von einem Atom}
Der gesamte Drehimpuls $\vec{j}$ eines einzelnen Hüllenelektrons setzt sich aus dem Spin $\vec{s}$ und dem Bahndrehimpuls $\vec{l}$ zusammen. Wie aus der klassischen Mechanik bekannt führt ein elektrischer Kreisstrom zu einem magnetischen Moment. Dieses Phänomen ist allerdings nur als Analogon der klassischen Mechanik zur Quantenmechanik zu sehen. Das Elektron hat ein magnetisches Moment, das vom Spin generiert wird, $\vec{\mu}_s$ und eines, das vom Bahndrehimpuls herrührt $\vec{\mu}_l$. Beide magnetischen Momente zeigen entgegengesetzt zum entsprechenden Drehimpuls und sind proportional zum Betrag der Drehimpulse. Der Betrag vom Spin und vom Bahndrehimpuls kann durch die diskreten Quantenzahlen $s$ und $l$ ausgedrückt werden.
\begin{align}
	| \vec{s} | &= \sqrt{s(s+1)} \hbar \\
	| \vec{l} | &= \sqrt{l(l+1)} \hbar
\end{align}
Die magnetischen Momente sind 
\begin{align}
	\vec{\mu}_s &= - g_s \mu_B  \sqrt{s(s+1)} \vec{e}_s \\
	\vec{\mu}_l & = - \mu_B  \sqrt{l(l+1)} \vec{e}_l 
\end{align}
mit den Einheitsvektoren $\vec{e}_s$ und $\vec{e}_l$, dem Landé-Faktor $g_s$ (für ein freies Elektron gilt~$g_s \approx 2$) und dem Bohrschen Magneton \\
\begin{equation}
	\mu_B =- \frac{e_0 \hbar}{2 m_0}  \quad .
\end{equation}
Gibt es nun mehr als ein Hüllenelektron, so wechselwirken die Spins und Bahndrehimpulse und addieren sich zu einem Gesamtdrehimpuls $\vec{J}$ aller Hüllenelektronen. Es lasse sich zwei Grenzfälle betrachten: Bei Atomen mit niedriger Kernladungszahl gilt die $LS$-Kopplung
\begin{align}
	\vec{J} = \vec{L} + \vec{S} = \sum_i \vec{l}_i + \sum_i \vec{s}_i \quad .
\end{align}
Wenn die Kernladungszahl groß ist, dominiert die Wechselwirkung zwischen den einzelnen Spins und Bahndrehimpulsen. Es gibt keinen Gesamtspin oder Gesamtbahndrehimpuls mehr, sonder nur noch einen Gesamtdrehimpuls. Man spricht von der $j$-$j$-Kopplung mit
\begin{align}
	\vec{J} = \sum_i \vec{j}_i = \sum_i \left( \vec{l}_i + \vec{s}_i \right) \quad .
\end{align}
Bei Atomen mit mittlerer Kernladungszahl existiert ein kontinuierlicher Übergang zwischen den Extremfällen. In diesem Versuch wird Cadmium untersucht. Cadmium hat die Kernladungszahl $Z = 48$ und fällt unter die $LS$-Kopplung. Um das gesamte magnetische Moment zu berechnen ist nur die nicht abgeschlossene Elektronenhülle relevant, da abgeschlossene Schalen keinen Gesamtdrehimpuls haben. Das gesamte magnetische Moment setzt sich also aus dem magnetischen Moment des Gesamtspins und des Gesamtbahndrehimpulses zusammen
\begin{align}
	\vec{\mu} = \vec{\mu}_L + \vec{\mu}_S  \quad .
\end{align}
Das magnetische Moment $\vec{\mu}$ ist wegen des Lande-Faktors des Spins nicht mehr parallel zum Gesamtdrehimpuls $\vec{J}$. Es ist aber die Projektion von $\vec{\mu}$ auf $\vec{J}$ relevant. Geometrische Überlegungen ergeben dann 
\begin{align}
	\vec{\mu}_J = \mu_B g \sqrt{J(J+1)} \vec{e}_J
\end{align}
mit dem Landé-Faktor
\begin{align}
g =	\frac{3J(J+1) + S(S+1) - L(L+1)}{2J(J+1)} \quad .
\end{align}
Hierbei ist $J$ die Quantenzahl des Gesamtdrehimpulses, $L$ die des Gesamtbahndrehimpulses und $S$ die des Gesamtspins.

\subsection{Normaler Zeeman-Effekt}\label{sec:normaler}
Beim normalen Zeeman-Effekt ist der Gesamtspin $S$ null. Der Effekt beschreibt folgendes: Ein magnetisches Moment in einem Magnetfeld $\vec{B}$ hat eine Energie, wenn das magnetische Moment parallel zu den Magnetfeldlinien ausgerichtet ist oder zumindest parallele Anteile hat. Der parallele Anteil des magnetischen Momentes kann nur diskrete Werte annehmen. Man spricht von der Richtungsquantelung. Die Orientierungsquantenzahl $m_J$ zu $J$ in Richtung des Magnetfeldes nimmt die Werte $J$, $J-1$ bis $-J$ an. Die Energie ist dann
\begin{align}
	E = - \vec{\mu}_J \cdot \vec{B} = m_J g \mu_B B \quad .
\end{align}
 Es erfolgt eine Aufspaltung der Energieniveaus mit den Energiedifferenzen
\begin{align}\label{eq:Energieverschiebung}
	dE = g\mu_B B \quad .
\end{align}
Der Landé-Faktor ist für alle Übergänge beim normalen Zeeman-Effekt identisch, da der Spin verschwindet und $J = L$ ist,  gilt 
\begin{align}
	g = 1 \quad .
\end{align}
Übergänge von Elektronen zwischen den Niveaus sind möglich, wenn die Übergänge die Auswahlregel $\Delta m_J = 0, \pm 1$ erfüllen. Zur Herleitung der Auswahlregel betrachtet man eine Superposition von zwei Lösungen der zeitabhängigen Schrödinger-Gleichung. Dieses Gesamtlösung $\psi(\vec{r}, t) = c_\alpha \psi_\alpha + c_\beta \psi_\beta$ hat einen Ortsanteil und einen Zeitanteil. Die Dichtefunktion $\int \psi^{*}\psi \dd V$ ist zeitabhängig. Das Elektron schwingt und erzeugt so einen elektrischen Dipol mit den charakteristischen Matrixelementen der Übergänge:
\begin{align}
	x_{\alpha \beta} = \int x \psi_\alpha^{*}\psi_\beta \dd V \quad
	y_{\alpha \beta} = \int y \psi_\alpha^{*}\psi_\beta \dd V \quad
	z_{\alpha \beta} = \int z \psi_\alpha^{*}\psi_\beta \dd V
\end{align}
Die Matrixelemente stehen in Zusammenhang mit dem Pointing-Vektor, der die Richtung des Energietransportes der Welle angibt. Nur für die Übergänge  $\Delta m_J = 0, \pm 1$ findet ein Energietransport statt.
\\Das Elektron gewinnt durch den Übergang in ein niedrigeres Energieniveaus Energie und strahlt  eine elektromagnetische Welle mit der Frequenz $\nu= \Delta E / h $ ab. Für $\Delta m_J = 0$ strahlt das Elektron linear polarisiertes Licht ($\pi$-Übergang) ab und für $\Delta m = \pm 1$ zirkular polarisiertes Licht ($\sigma_{\pm}$-Übergang). 
Die Übergänge unterschiedlicher Polarisation können nur in bestimmten Beobachtungsrichtungen zum Magnetfeld gesehen werden. Das linear polarisierte Licht ist sichtbar, wenn man den Anteil beobachtet, der parallel zu den Magnetfeldlinien steht und den zirkular polarisierten Anteil entsprechend bei der Betrachtung des senkrechten Anteils, siehe Abb. \ref{fig:abb4}. Der Grund dafür ist die Ausrichtung des Dipols zum Magnetfeld.
\begin{figure}
	\includegraphics[width = 0.4\textwidth]{Abb4.pdf}
	\caption[Polarisationsrichtung]{Aufspaltung des Energieniveaus für $J=1$ in Abhängigkeit der Beobachtungsrichtung des ausgestrahlten Lichtes \cite{\V}}
	\label{fig:abb4}
	\end{figure}
\subsubsection{Cadmium -- rote Spektrallinie $\lambda = \SI{643.8}{\nano\meter}$}
Bei dem Übergang von dem Niveau $\ce{^1D_2}$ auf das Niveau $\ce{^1P_1}$ entsteht eine rote Spektrallinie mit einer Wellenlänge von $\lambda = \SI{643.8}{\nano\meter}$. Die Quantenzahlen der Niveaus sind:
\begin{align}
	&\ce{^1D_2} : \quad L=1, J =1, S =0 \notag \\
	&\ce{^1P_1}: \quad L = 2, J = 2, S = 0 \notag
\end{align}	
Der Landè-Faktor ist immer $g=1$. Es sind neun verschiedene Übergänge möglich (siehe Abbildung \ref{fig:mausi1} und Tabelle \ref{tab:g}). Es gilt
\begin{align}
	\Delta E &= \left( m_{J,1} g_1(L_1, S_1, J_1) - m_{J,2} g_2(L_2, S_2, J_2)  \right) \mu_B B \notag \\
	& =g_{12} \mu_B  B
\end{align}
mit
\begin{align}
	g_{12} = m_{J,1} g_1(L_1, S_1, J_1) - m_{J,2} g_2(L_2, S_2, J_2)  \quad .
\end{align}
\begin{figure}
	\includegraphics[width = 0.4\textwidth]{Mausi1.png}
	\caption[Aufspaltung von $\ce{^1D_2}$ auf $\ce{^1P_1}$]{Aufspaltung mögliche Übergänge  von Niveau $\ce{^1D_2}$ auf das Niveau $\ce{^1P_1}$. \cite{V27_mausi}}
	\label{fig:mausi1}
\end{figure}
\begin{table}
	\begin{tabular}{cccccc}
		& $m_1$ & $g_1$ & $m_2$ & $g_2$ & $g_{12}$ \\
		\hline
		$\sigma_-$ & -1 & 1& 0 & 0 & -1 \\
		& 0 & 1 & 1 & 1 & -1 \\
		& 1 & 1 & 2 & 1 & -1 \\
		\hline
		$\pi$ & -1 & 1 & -1 & 1 & 0 \\
		& 0 & 1 & 0 & 1 & 0 \\
		& 1 & 1 & 1 & 1 & 0 \\
		\hline
		$\sigma_+$  & -1 & 1 & 0 & 1 & 1 \\
		& 0 & 1 & -1 & 1 & 1 \\
		& 1 & 1 & -2 & 1 & 1
	\end{tabular}
	\caption[Normaler Zeeman-Effekt $g_{12}$]{Berechnung der Faktoren $g_{12}$ bei dem Übergang von Niveau $\ce{^1D_2}$ auf das Niveau $\ce{^1P_1}$.}
	\label{tab:g}
\end{table}

\subsection{Anormaler Zeeman-Effekt}
Der anormale Zeeman-Effekt unterscheidet sich vom normalen Zeeman-Effekt dadurch, dass der Gesamtspin $S$ von null verschieden ist. Der anormale Zeeman-Effekt tritt häufiger auf als der normale Zeeman-Effekt. Die Auswahlregel für die Übergänge $\Delta m_J = 0, \pm 1$ gilt weiterhin und auch die Polarisation der emittieren Welle ist genau wie beim normalen Zeeman-Effekt (siehe Kaptiel \ref{sec:normaler}). Die Energiedifferenzen zwischen den Niveaus sind hingegen nicht mehr äquidistant, da der Landé-Faktor der einzelnen Niveaus unterschiedlich ist.
\subsubsection{Cadmium -- blaue Spektrallinie $\lambda = \si{480}{\nano\meter}$}
Der Übergang von dem Niveau $\ce{^3P_1}$ auf das Niveau $\ce{^3S_1}$ erzeugt eine blaue Spektrallinie mit einer Wellenlänge von $\lambda = \si{480}{\nano\meter}$. Die Quantenzahlen der Niveaus sind
\begin{align}
	&\ce{^3P_1} : \quad L=1, J =1, S =1 \notag \\
	&\ce{^3S_1}: \quad L = 0, J = 1, S = 1 \notag
 \end{align}	
Die Übergänge sind in Abbildung \ref{fig:mausi2} dargestellt und die Faktoren $g_{12}$ sind in Tabelle~\ref{tab:g12} zu finden.

\begin{figure}
	\includegraphics[width = 0.4\textwidth]{Mausi2.png}
	\caption{Mögliche Übergänge  von Niveau $\ce{^3P_1}$ auf das Niveau $\ce{^3S_1}$. \cite{V27_mausi}}
	\label{fig:mausi2}
\end{figure}

\begin{table}
\begin{tabular}{cccccc}
	& $m_1$ & $g_1$ & $m_2$ & $g_2$ & $g_{12}$ \\
	\hline
	$\sigma_-$ & -1 & 2& 0 & 1.5 & -2 \\
	& 0 & 2 & 1 & 1.5 & -1.5 \\
	\hline
	$\pi$ & -1 & 2 & -1 & 1.5 & -0.5 \\
	& 0 & 2 & 0 & 1.5 & 0 \\
	& 1 & 2 & 1 & 1.5 & 0.5 \\
	\hline
	$\sigma_+$ & 0 & 2 & -1 & 1.5 & 1.5 \\
	& 1 & 2 & 0 & 1.5 & 2
\end{tabular}
\caption[Anormaler Zeeman-Effekt $g_{12}$]{Berechnung der Faktoren $g_{12}$ bei dem Übergang von Niveau  $\ce{^3P_1}$ auf das Niveau $\ce{^3S_1}$.}
\label{tab:g12}
\end{table}





\clearpage


\section{Aufbau und Ablauf des Experiments}
In diesem Kapitel wird zunächst die Hochfrequenzmethode zur Bestimmung der Energiedifferenz erklärt und dann das konkrete Vorgehen bei der Versuchsdurchführung.

\subsection{Hochfrequenzmethode}
Nachdem sich die Besetzungsinversion eingestellt hat, gibt es zwei verschiedene Arten, wie die Elektronen zurück in den Grundzustand gelangen können. Dies sind die spontane Emission und die induzierte Emission, die durch anregende Photonen mit einer Energie, die genau die der Energiedifferenz entspricht, erfolgt
\begin{align}\label{eq:Resonanz}
	h\nu = g_F\mu_\text{B}B_m \quad .
\end{align}
Welche Art der Emission dominiert hängt im wesentlichen von der Frequenz des Abstrahlvorgangs ab. In unserem Fall treten fast ausschließlich induzierte Emissionen auf. Um die Breite der Energielücke zu bestimmen sucht man die Resonanzstelle. \\
\begin{figure}[H]
	\centering
	\includegraphics[width=0.6\textwidth]{Abb10.pdf}
	\caption[Transparenz]{Transparenz des Alkali-Gases in einem von außen angelegtem Magnetfeld, um die Resonanzfrequenz zu treffen \cite{\V}}
	\label{fig:transparenz}
\end{figure}
Die Resonanzstelle kann gut durch die Transparenz des Gases gefunden werden (siehe Abb. \ref{fig:transparenz}). Das Gas wird permanent mit rechtszirkular polarisiertem Licht bestrahlt. Ab dem Moment wenn die Besetzungsinversion einsetzt, ist das Gas transparent. Wenn nun die induzierte Emission eintritt, ist das Grundniveau wieder besetzt und das einfallende Licht hebt die Elektronen auf einen höheren Zustand. Das Gas ist nicht mehr vollständig transparent. Das Gas erreicht diesen Zustand jedoch bald wieder.\\
Der Versuchsaufbau, um die Resonanzen zu finden ist in Abb. \ref{fig:aufbau} dargestellt. Die Spektrallampe liefert das Licht, um die Elektronen anzuregen. Das Licht wird in einem Polarisator rechtszirkular polarisiert und mit einer Sammellinse auf das Alkali-Gas aus Rubidium-85 und Rubidium-87 Isotopen fokussiert. Das erhitzte Alkali-Gas befindet sich im thermodynamischen Gleichgewicht. Der Kolben ist von drei Holmholtz-Spulen umgeben und einer Spule, die die Hochfrequenz generiert. Die Frequenz wird immer auf einen festen Wert zwischen \si{100}{\kilo\hertz} und \si{1}{\mega\hertz} eingestellt und das Magnetfeld fährt einen Bereich kontinuierlich durch (Modulationsfeldspule). Auf der anderen Seite des Kolbens ist eine Diode, die an ein Oszilloskop angeschlossen ist. 
\begin{figure}[H]
	\centering
	\includegraphics[width=\textwidth]{Abb13.pdf}
	\caption{Versuchsaufbau \cite{\V}}
	\label{fig:aufbau}
\end{figure}


\subsection{Durchführung}
Insgesamt wird eine Messreihe und ein Bild aufgenommen. In der Messreihe werden die Frequenzen und Magnetfelder an der Resonanzstelle der beiden verschiedenen Isotope aufgenommen. Das Bild zeigt den Spannungsverlauf am Oszilloskop, der die typische Transparenzkurve darstellt. Die einzelnen Schritte der Vorbereitung und der Messung sind folgende:
\begin{enumerate}
	\item Der Strahlengang wird durch die Sammellinsen so aufgebaut, dass das die Diode eine maximale Spannung ausgibt. 
	\item Das Magnetfeld wird berücksichtigt. Die horizontale Komponente wird durch Drehen des gesamten Aufbaus verändert und die vertikale durch die vertikale Helmholtz-Spule. Beides wird so eingestellt, dass die ansteigende Flanke der Transparenz-Kurve möglichst steil ist. 
	\item Für den Frequenzbereich \si{100}{\kilo\hertz} bis \si{1}{\mega\hertz} wird in Abständen von \si{100}{\kilo\hertz} die Resonanzstelle beider Isotope gesucht. Dafür verwendet man die Modulationsfeldspule. Reicht der Bereich der Modulationsfeldspule nicht aus, wird zusätzlich ein Magnetfeld durch die horizontale Helmhotz-Spule angelegt.
	\item Ein Bild eines typischen Signalverlaufs soll aufgenommen werden.
\end{enumerate}



\clearpage


\section{Auswertung}
\subsection{Polarisation}
Die Winkel mit den zugehörigen Intensitäten sind in Tabelle \ref{tab:Polarisation} aufgetragen. Mit Hilfe der Funktion \textit{curve\_fit} von Python und einem First Guess von
\begin{align*}
	I_\text{guess} = \SI{100}{\micro\ampere}\sin^2(\varphi + 0.4)
\end{align*}
wird die Funktion
\begin{align*}
	I_\text{fit} = I_0\sin^2\left(\omega\varphi + \varphi_0\right)
\end{align*}
an die Messwerte gefittet. Dabei ergeben sich die Parameter
\begin{align}
	I_0 &= \SI{94+-1e-6}{\micro\ampere}
 \\
	\omega &= \SI{1.013+-0.006}{}
 \\
	\varphi_0 &= \SI{0.44+-0.02}{}
 \ .
\end{align}
Die Messwerte, der First Guess und die gefittete Funktion sind in Abbildung \ref{fig:fitPol} zu sehen. Die Maxima befinden sich bei \SI{0.3\pi}{}
- bzw. $1.3\pi$, d.h. der Laser war in diese Richtung polarisiert.
\begin{figure}[h!]
	\centering
	\includegraphics[width=.6\textwidth]{Fit_Polarisation.png}
	\caption{Fit zur Bestimmung der Polarisationsrichtung}
	\label{fig:fitPol}
\end{figure}
\begin{table}
    \centering
    \caption{Intensität in Abhängigkeit des Winkels des Polarisators}
    \label{tab:Polarisation}
    \sisetup{parse-numbers=false}
    \begin{tabular}{
	S[table-format=1.1]
	S[table-format=3.1]
	}
	\toprule
	{$\varphi$}		& {$I \ \mathrm{in} \ \si{\micro\ampere}$}		\\ 
	\midrule
    \input{build/tablePolarisation.tex}
    \bottomrule
    \end{tabular}
    \end{table}

\clearpage
 \subsection{Berechnung der Wellenlänge}
 Die Wellenlänge des HeNe-Lasers wird mit Hilfe eines Interferenzbildes berechnet, das durch die Beugung am Gitter entsteht. Jeweils rechts und links vom Hauptmaximum (Maximum 0-ter Ordnung) werden die Abstände $a$ der Maxima $n$-ter Ordnung zum Hauptmaximum gemessen. Die Wellenlänge $\lambda$ lässt sich dann mit dem Abstand $b$ zwischen Gitter und Schirm berechnen nach:
 
 \begin{equation}
 \lambda = \frac{g \cdot \sin{\phi_n}}{n} \quad .
 \end{equation}
 
Dabei ist $g$ die Gitterkonstante und $\phi_n$ der Winkel zum $n$-ten Maximum.
Es gilt
\begin{align}
\phi_n = \arctan{\left( \frac{a}{b} \right)} \quad \textrm{,}\\
g = \SI{100}{\per\milli\meter} \quad \textrm{,}\\
b = \SI{104}{\centi\meter} \quad .
\end{align}
 
 Die Wellenlänge für das einzelne Maximum ist jeweils aus dem Mittelwert des rechten und linken Abstandes berechnet (siehe Tabelle~\ref{tab:Wellenlaenge}). Daraus ergibt sich eine Wellenlänge von
\begin{equation}
\lambda = \SI{0.0000000000000006370+-0.0000000000000000014}{\nano\meter}
 \quad .
\end{equation}
\begin{table}
    \centering
    \caption{Berechnung der Wellenlänge $\lambda$ durch die Abstände der Interferenzmaxima}
    \label{tab:Wellenlaenge}
    \sisetup{parse-numbers=false}
    \begin{tabular}{
	S[table-format=1.0]
	S[table-format=2.2]
	S[table-format=2.2]
	S[table-format=3.2]
	}
	\toprule
	{Ordnung Maximum }		& {Abstand $a$ rechts in \si{\centi\meter}}		& 
	{Abstand $a$ links in \si{\centi\meter}}		& {$\lambda$ in $\si{\nano\meter}$}		\\ 
	\midrule
    \input{build/tab_wellenlaenge.tex}
    \bottomrule
    \end{tabular}
    \end{table}

\clearpage
\begin{figure}[h!]
	\centering
	\includegraphics[width=.6\textwidth]{FitCurved.png}
	\caption{Fit zur Überprüfung der Stabilitätsbedingung bei zwei konkaven Spiegeln}
	\label{fig:fitcurved}
\end{figure}
\begin{figure}[h!]
	\centering
	\includegraphics[width=.6\textwidth]{FitFlat.png}
	\caption{Fit zur Überprüfung der Stabilitätsbedingung bei einem gekrümmten und einem flachen Spiegel}
	\label{fig:fitflat}
\end{figure}
\clearpage
\subsection{Intensität der TEM$_{00}$- und TEM$_{01}$-Mode}
\subsubsection{TEM$_{00}$-Mode}

Fitparameter sind die maximale Amplitude $I_0$, die Breite des Laser-Strahls $\omega$, nachdem dieser die Zerstreuungslinse durchlaufen hat, und die Verschiebung des Maximums entlang der x-Achse $x_0$. Die Messdaten sind in Tabelle REF zu finden und der zugehörige Plot ist Abbildung \ref{fig:TEM_00}.

\begin{align}
	I_0 =  \SI{446+-23}{\nano\ampere}
\\
	\omega =  \SI{-11.2+-0.7}{\milli\meter}
 \\
	x_0 = \SI{3.55+-0.34}{\milli\meter}

\end{align}
	
 \begin{table}
    \centering
    \caption{Intensität der TEM$_{00}$-Mode entlang der x-Achse}
    \label{tab:TEM_00}
    \sisetup{parse-numbers=false}
    \begin{tabular}{
	S[table-format=3.1]
	S[table-format=3.2]
	}
	\toprule
	{I in \si{\nano\ampere}	}	& {x in \si{\milli\meter}}		\\ 
	\midrule
    \input{build/tab_TEM_00.tex}
    \bottomrule
    \end{tabular}
    \end{table}



\begin{figure}[h!]
	\centering
	\includegraphics[width=.6\textwidth]{build/TEM_00.png}
	\caption{Intensität der TEM$_{00}$-Mode entlang der x-Achse gemessen}
	\label{fig:TEM_00}
\end{figure} 

blblblbl
\clearpage


\section{Diskussion}
Tabelle \ref{tab:Ergebnisse} zeigt die aus der Messung gewonnenen Ergebnisse und vergleicht sie mit Literaturwerten.

\begin{table}[h!]
	\centering
	\caption{Aus der Messung berechnete Werte der Flussdichte $B_\text{Erde}$ des Erdmagnetfeldes und des Landé-Faktors $g$ verglichen mit Literaturwerten aus \cite{BErde} und \cite{gFaktor}}
	\label{tab:Ergebnisse}
	\begin{tabular}{l|cc}
		& $B_\text{Erde}$ & $g$ \\
		\hline
		Messung & \SI{0.047+-0.003}{\milli\tesla}
 & \SI{2.04+-0.03}{}
 \\
		Literatur & \SI{0.049}{\micro\tesla} & \SI{2.0023(0)}{} \\
		\hline
		Abweichung & -\SI{4.1}{\%} & +\SI{1.7}{\%}
	\end{tabular}
\end{table}

Der Literaturwert für die Feldstärke des Erdmagnetfeldes liegt im Toleranzbereich des Messwertes. Es könnte vermutet werden, dass die kleine Abweichung nach unten daran liegt, dass die Apparatur nicht perfekt in Richtung des Erdfeldes ausgerichtet war, allerdings sind Messungen des Erdmagnetfeldes generell sehr schwierig, sodass das reine Spekulation ist und der erreichte Wert als sehr gut zu bewerten ist. \\
Der berechnete Landé-Faktor weicht nur wenig vom Literaturwert ab, wobei auch die Unsicherheit sehr klein ist. \\
Allerdings gilt zu Beachten, dass einige Quellen von Messunsicherheiten in der Rechnung ignoriert wurden. So war das Extremum, das mit Hilfe des Kondensators und des Widerstands eingestellt wird, bei \SI{10}{\mega\hertz} und \SI{30}{\mega\hertz} vermutlich nur lokal und nicht global. Zudem sind die meisten der aufgenommenen Kurven nicht symmetrisch, das liegt daran, dass die Abstimmung der Brückenschaltung nicht optimal war, sodass die Kurven nicht reine Resonanzkurven sind, sondern Anteile von Dispersionskurven haben. Besonders gut ist dieser Effekt bei den Kurven in \ref{fig:Scan10} und \ref{fig:Scan20} zu sehen. Zusätzlich dazu kann es bei elektronischen Schaltungen immer zu Rückkopplungseffekten kommen, die die Messung verfälschen.


\clearpage
\listoftodos
\listoffigures
\listoftables
\clearpage
\nocite{\V}
\printbibliography[title = Literaturverzeichnis]

\end{document}
