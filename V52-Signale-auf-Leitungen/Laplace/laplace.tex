Die Signalverläufe geben Aufschluss über die Leitungskonstanten der Kabel. Wie in Kapitel \ref{seq:laplace} gezeigt, entstehen unterschiedliche Spannungsverläufe für unterschiedliche Ausgangsimpedanzen. Das geschlossene Kabel entspricht einer Induktivität und das offene Kabel einer Kapazität. Als Impedanz $Z_0$ ist für das vermessene Kabel ist
\begin{equation}
	Z_0 = \si{50}{\ohm}
\end{equation}
angegeben. Der Spannungsverlauf wird mit einem Oszilloskop aufgezeichnet und die Werte an den theoretischen Verlauf der Kurve gefittet.

\subsubsection{Kabel mit geschlossenem Ende}
Der Signalverlauf ist in Abb. \ref{fig:oszi_geschlossen} dargestellt. Erwartet wird eine Flanke der Form
\begin{align}
			U(t) &= 2U_0\left(1 - \frac{Z_0}{R+Z_0} + \frac{Z_0}{R+Z_0}\exp\left(-\frac{R+Z_0}{L}t\right)\right) \\
			&= 2 a \left(1- b + b \exp(-c (t+d))   \right) + e \quad .
\end{align}
Es ist nur mit einer geeigneten Wahl der Startparameter möglich, einen guten Fit zu bekommen. Leider sind die so erhaltenen Abweichungen der Parameter dann unrealistisch und werden deswegen nicht weiter betrachtet. Die Funktion ist in Abb. \ref{fig:fit_geschlossen} dargestellt. Die verwendeten Parameter sind:
\begin{align}
	a &= \SI{-6.7339}{\volt}
 \\
	b &= \SI{0.9872}{}
 \\
	c &= \SI{0.0089}{\ohm\per\henry}
 \\
	d &= \SI{483.2099}{\nano\second}
 \\
	e &= \SI{12.5272}{\volt}
 
\end{align}
Aus den Parametern können der Widerstandsbelag $R$ und der Induktivitätsbelag $L$ des Kabels berechnet werden. Es muss beachtet werden, dass die Zeitskala Nanosekunden ist.
\begin{align}
	R &= \frac{Z_0}{b} - Z_0 = \SI{0.6507}{\ohm}
 \\
	L &= \frac{R + Z_0}{ c \cdot 10^9} = \SI{5.6828}{\micro\henry}

\end{align}



\begin{figure}
	\centering
	\includegraphics[width=0.7\textwidth]{Laplace/geschlossen.pdf}
\caption{Spannungsverlauf bei dem Kabel mit geschlossenem Ende}
\label{fig:oszi_geschlossen}
\end{figure}

\begin{figure}
	\centering
	\includegraphics[width=0.7\textwidth]{Laplace/geschlossenes_ende.pdf}
	\caption{Fitfunktion an die fallende Flanke des Signalverlaufs bei dem Kabel mit geschlossenem Ende}
\label{fig:fit_geschlossen}
\end{figure}

\clearpage
\subsubsection{Kabel mit offenem Ende}
Der Spannungsverlauf des offenen Kabels ist in  Abb. \ref{fig:oszi_offen} zu sehen. Der Verlauf folgt der Funktionsvorschrift
\begin{align}
	U_0(t) &= 2U_0\left(1 -  \frac{Z_0}{R+Z_0} \exp \left( - \frac{1}{C(R+Z_0)} t\right) \right)\\
	&=  2 a (1 - b \exp(-c (x-d))) + e \quad .
\end{align}
Wie schon bei dem Kabel mit geschlossenem Ende müssen die Startwerte für die Parameter des Fits sehr exakt gewählt werden. Auch die ausgegebenen Abweichungen sind abermals nicht brauchbar. Der Plot ist in Abb. \ref{fig:fit_offen} dargestellt. Die Parameter sind:

\begin{align}
		a &= \SI{4.1028}{\volt}
 \\
		b &= \SI{0.6497}{}
 \\
		c &= \SI{0.0132}{\per\farad\per\ohm}
 \\
		d &= \SI{-466.8082}{\nano\second}
 \\
		e &= \SI{-10.8431}{\volt}
 
\end{align}
Es lassen sich die Kapazität $C$ des Kabels ebenso die der Widerstand $R$ aus den Parametern berechnen 
\begin{align}
	R &= \frac{Z_0}{b} - Z_0 = \SI{26.9529}{\ohm}
 \\
	C &= \frac{1}{ c (r + Z_0) \cdot 10^9} = \SI{985.9779}{\pico\farad}
 \quad.
\end{align}

\clearpage

\begin{figure}
	\centering
	\includegraphics[width=0.7\textwidth]{Laplace/offen.pdf}
	\caption{Spannungsverlauf bei dem Kabel mit offenem Ende}
	\label{fig:oszi_offen}
\end{figure}

\begin{figure}
	\centering
	\includegraphics[width=0.7\textwidth]{Laplace/offenes_ende.pdf}
	\caption{Fitfunktion an die fallende Flanke des Signalverlaufs bei dem Kabel mit offenem Ende}
	\label{fig:fit_offen}
\end{figure}
