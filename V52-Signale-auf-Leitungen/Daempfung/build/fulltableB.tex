\begin{table}
    \centering
    \caption{Frequenz $\omega$ und Betrag der Amplitude $P$ der Peaks in der FFT für das kurze und das lange Kabel, sowie mit \eqref{eq:DaempfungB} berechnete Dämpfung}
    \label{tab:DaempfungWerteB}
    \sisetup{parse-numbers=false}
    \begin{tabular}{
	S[table-format=2.3]
	S[table-format=2.2]
	S[table-format=2.3]
	S[table-format=2.2]
	S[table-format=2.2]
	}
	\toprule
	{$\omega_\text{Kurz} \ \mathrm{in} \ \si{\mega\hertz}$}		& {$|P_\text{Kurz}| \ \mathrm{in} \ \si{\deci\bel}$}		& 
	{$\omega_\text{Lang} \ \mathrm{in} \ \si{\mega\hertz}$}		& {$|P_\text{Lang}| \ \mathrm{in} \ \si{\deci\bel}$}		& 
	{$\alpha \ \mathrm{in} \ \si{\per\kilo\meter}$}		\\ 
	\midrule
    0.644  & 0.59  & 0.644  & 0.59  & -0.00 \\
1.963  & 10.19 & 1.963  & 10.19 & -0.00 \\
3.252  & 14.99 & 3.252  & 14.99 & -0.00 \\
4.571  & 17.79 & 4.541  & 18.59 & 1.84  \\
5.860  & 19.79 & 5.860  & 20.99 & 2.76  \\
7.179  & 21.79 & 7.179  & 22.99 & 2.76  \\
8.468  & 23.39 & 8.468  & 24.59 & 2.76  \\
9.787  & 24.59 & 9.787  & 25.79 & 2.76  \\
11.106 & 25.79 & 11.075 & 27.39 & 3.68  \\
12.395 & 26.59 & 12.395 & 28.59 & 4.61  \\
13.683 & 27.79 & 13.683 & 29.79 & 4.61  \\
15.002 & 28.59 & 15.002 & 30.99 & 5.53  \\
16.322 & 29.79 & 16.291 & 31.79 & 4.61  \\
17.610 & 30.19 & 17.610 & 32.19 & 4.61  \\
18.929 & 31.39 & 18.929 & 33.39 & 4.61  \\
20.218 & 31.79 & 20.218 & 34.19 & 5.53  \\
21.537 & 32.19 & 21.537 & 34.59 & 5.53  \\
22.826 & 33.39 & 22.856 & 35.79 & 5.53  \\
24.145 & 34.59 & 24.114 & 36.99 & 5.53  \\
25.433 & 34.99 & 25.433 & 37.39 & 5.53  \\

    \bottomrule
    \end{tabular}
    \end{table}
