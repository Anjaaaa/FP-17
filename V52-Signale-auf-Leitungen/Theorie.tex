\subsection{Verlustbehaftete Leitungen}
Um die elektrischen Eigenschaften von Bauteilen zu Modellieren wird üblicherweise ein Ersatzschaltbild verwendet. Abbildung \ref{fig:Ersatzschaltbild} zeigt die Ersatzschaltbilder für eine verlustfreie und eine verlustbehaftete Leitung. Die Bausteine werden in diesem Fall als \textit{Beläge} bezeichnet: Kapazitätsbelag $C$, Induktivitätsbelag $L$, ohmscher Belag $R$, Querleitfähigkeitsbelag $G$. Es wird zudem zwischen Längsspannungsverlust (an $R$) und Querstromverlust (an $G$) unterschieden, wobei die ohmschen bzw. Längsspannungsverluste meist überwiegen.
\begin{figure}[h!]
	\centering
	\begin{subfigure}[t]{0.35\textwidth}
	\centering
\begin{circuitikz}
	\draw (-.5, 0)
	to[american inductor, l = $L$] (2.5, 0)
	to (3.5, 0);
	\draw (-.5, -2)
	to (3.5,-2);
	\draw (2.5,0)
	to[C, l = $C$] (2.5,-2);	
\end{circuitikz}
\subcaption{}
\label{fig:OhneVerlust}
\end{subfigure}
\begin{subfigure}[t]{0.55\textwidth}
	\centering
\begin{circuitikz}
	\draw (-.5, 0)
	to[R, l = $R$] (2.5,0)
	to[american inductor, l = $L$] (3.5, 0)
	to (7, 0);
	\draw (-.5, -2)
	to (7,-2);
	\draw (4.5,0)
	to[R, l = $G$] (4.5,-2);
	\draw (6,0)
	to[C, l = $C$] (6,-2);	
\end{circuitikz}
\subcaption{}
\label{fig:MitVerlust}
\end{subfigure}
	\caption{Ersatzschaltbilder einer verlustfreien (\ref{fig:OhneVerlust}) und einer verlustbehafteten Leitung (\ref{fig:MitVerlust})}
	\label{fig:Ersatzschaltbild}
\end{figure} \\
Mit Hilfe des Ersatzschaltbildes für die verlustbehaftete Leitung kann die Telegraphengleichung
\begin{align}
	\pdv[2]{U}{t} = LC\pdv[2]{U}{x} + (LG + RC)\pdv{U}{x} + RGU
\end{align}
für die Spannung $U(x,t)$ abgeleitet werden. Ihre Lösung ist
\begin{align}
	U(x,t) = U_0 \exp(-\gamma x)\exp(i\omega t) \quad ,
\end{align}
hierbei ist $\omega$ die Frequenz der Spannung und $\gamma = \alpha + ik = \sqrt{(R+i\omega L)(G+i\omega C)}$ die Ausbreitungskonstante. Sie enthält den Dämpfungsbelag $\alpha$ und den Phasenbelag $k$ (die Wellenzahl).